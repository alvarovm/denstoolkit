Usage:

	dtkmomd wf?name [option [value(s)]] ... [option [value(s)]]

Where wf?name is the input wfx(wfn) name, and options can be:
Where wf?name is the input wfx(wfn) name, and options can be:

  -0 x y z  	Evaluate the momentum density at the point (x,y,z).
            	  Here x,y, and z are numbers. No file is created.
  -1  x     	Evaluate the momentum density on a line. "x" sets the coordinate
            	  direction, and can take the values x, y or z.
            	  Here x is a character. It creates a *.dat file.
  -2  xy    	Evaluate the momentum density on a plane. "xy" sets the plane of
            	  interest, and can take the values xy,xz,yz.
            	  Here xy are two characters. It creates a *.tsv file.
  -3        	Evaluate the momentum density on a cube.
            	  It creates a *.cub file.
  -n  dim   	Set the number of points for the cub/tsv/dat file per direction
  -o outname	Set the output file name.
            	  (If not given the program will create one out of
            	  the input name; if given, the dat/tsv/cub/gnp/pdf files will
            	  use this name as well --but different extension--).
  -p prop   	Choose the property to be computed. prop is a character,
            	  which can be (d is the default value):
            	     d Density (momentum density)
            	     K Kinetic Energy Density (in momentum space)
  -P     	Create a plot using gnuplot. (Only works with options -1 or -2)
  -k     	Keeps the *.gnp file to be used later by gnuplot.
  -v     	Verbose (display extra information, usually output from third-
         	  party sofware such as gnuplot, etc.)
  -z     	Compress the tsv file using gzip (which must be intalled
         	   in your system).
  -V        	Displays the version of this program.
  -h     	Display the help menu.

  --help    		Same as -h
  --version 		Same as -V
********************************************************************************
  Note that the following programs must be properly installed in your system:
                                    gnuplot
                                    epstool
                                    epstopdf
                                      gzip
********************************************************************************
