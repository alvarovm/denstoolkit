

Usage:

	dtkbpdens wf?name [option [value(s)]] ... [option [value(s)]]

Where wf?name is the input wfx(wfn) name, and options can be:
Where wf?name is the input wfx(wfn) name, and options can be:

  -a a1 a2  	Define the atoms  (a1,a2) used to define bond path/line.
            	  If this option is not activated, the program will 
            	  set a1=1, a2=2.
            	  Note: if the *.wfn (*.wfx) file has only one atom
            	  the program will exit and no output will be generated.
  -L        	Calculate the field upon the straight line that joins the atoms
            	  instead of upon the bond path.
  -n  dim   	Set the number of points for the dat file.
            	  Note: for the bond path you may want to look for a good 
            	  combination of n and the number "step" given in option -s,
            	  since the number of points in the bond path will be mainly 
            	  governed by step.
  -o outname	Set the output file name.
            	  (If not given the program will create one out of
            	  the input name; if given, the dat, gnp and (eps)pdf files will
            	  use this name as well --but different extension--).
  -p prop   	Choose the property to be computed. prop is a character,
            	  which can be (d is the default value): 
            		d (Density)
            		g (Magnitude of the Gradient of the Density)
            		l (Laplacian of density)
            		K (Kinetic Energy Density K)
            		G (Kinetic Energy Density G)
            		E (Electron Localization Function -ELF-)
            		L (Localized Orbital Locator -LOL-)
            		M (Magnitude of the Gradient of LOL)
            		P (Magnitude of Localized Electrons Detector -LED-)
            		r (Region of Slow Electrons -RoSE-)
            		s (Reduced Density Gradient -s-)
            		S (Shannon Entropy Density)
            		V (Molecular Electrostatic Potential)
            		u (Scalar Custom Field)
  -s step   	Set the stepsize for the bond path to be 'step'.
            	  Default value: 0.03

  -P     	Create a plot using gnuplot.
  -l     	Show labels of atoms (those set in option -a) in the plot.

  -v     	Verbose (display extra information, usually output from third-
         	  party sofware such as gnuplot, etc.)
  -z     	Compress the dat file using gzip (which must be intalled
         	   in your system).

  -h     	Display the help menu.
  -V     	Displays the version of this program.
  --help    		Same as -h
  --version 		Same as -V

--------------------------------------------------------------------------------
            	The format of the dat file is:
            	      L  X  Y  Z  V
            	  where L is the value of the parameter that maps the bond
            	  path to a line; X, Y, and Z are the actual spatial coordinates
            	  of each point in the bond path; and V is the value of the
            	  chosen field at the point (X,Y,Z) ---see option -p.
--------------------------------------------------------------------------------
********************************************************************************
  Note that the following programs must be properly installed in your system:
                                    gnuplot
                                    epstopdf
                                      gzip
********************************************************************************
