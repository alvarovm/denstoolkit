%%                     This source code is part of
%% 
%%                   D  E  N  S  T  O  O  L  K  I  T
%% 
%%                          VERSION: 1.1.1
%% 
%%              Contributors: Juan Manuel Solano-Altamirano
%%                            Julio Manuel Hernandez-Perez
%%         Copyright (c) 2013-2015, Juan Manuel Solano-Altamirano
%%                                  <jmsolanoalt@gmail.com>
%% 
%%  -------------------------------------------------------------------
%%  Copyright (c) 2013-2015 Juan Manuel Solano Altamirano.
%%  Permission is granted to copy, distribute and/or modify this document
%%  under the terms of the GNU Free Documentation License, Version 1.3
%%  or any later version published by the Free Software Foundation;
%%  with no Invariant Sections, no Front-Cover Texts, and no Back-Cover Texts.
%%  A copy of the license is included in the section entitled "GNU
%%  Free Documentation License".
%%  ---------------------------------------------------------------------
%% 
%%  If you want to redistribute modifications of the suite, please
%%  consider to include your modifications in our official release.
%%  We will be pleased to consider the inclusion of your contributions.
%%  within the official distribution. Please keep in mind that
%%  scientific software is very special, and version control is 
%%  crucial for tracing bugs. If in despite of this you distribute
%%  your modified version, please do not call it DensToolKit.
%% 
%%  If you find DensToolKit useful, we humbly ask that you cite
%%  the paper(s) on the package --- you can find them on the top
%%  README file.
%% 
%% ************************************************************************
%%%%


%**********************************************************************************************
%**********************************************************************************************
\chapter{Topological analysis}\label{sec:topolanalysis}
%**********************************************************************************************
%**********************************************************************************************

In addition to evaluate density fields on different grids, one of the purposes of \DTK{} is to seek the critical points of the electron density of a molecule, as proposed in Quantum Theory of Atoms in Molecules (QTAIM). By critical point, we mean any point $\boldsymbol{r}_*$ that accomplishes
%
\begin{equation}
   \nabla F(\boldsymbol{r}_*)=\boldsymbol{0},
\end{equation}
%
where $F$ is any density field. The critical points can be classified according to the sign of the Hessian eigenvalues at $\boldsymbol{r}_*$, as we explain below. 

In version \dtkversion, we have implemented the seek of all electron density critical points, including bond paths, and some of the critical points of LOL.

The development of this part of \DTK{} will be active during the upcoming years, and it is expected to grow as that of the needs of our research group. Obviously, for the interested developer, any contribution
will be welcome.

%==============================================================================================
\section{Classification of critical points}
%==============================================================================================

A critical point can be classified by means of the Hessian eigenvalues at the critical point.
Since we will be interested only in three dimensional spaces, and on non-degenerated Hessians,
the Hessian will always have three eigenvalues. Therefore, we can find four combinations of
the eigenvalues' signs. We will follow the QTAIM notation, wherein a critical point is
denoted as a couple $(3,s)$; here ``3'' stands for the three eigenvalues of the Hessian, and
$s=n_+-n_-$, where $n_+$ ($n_-$) is the number of positive(negative) eigenvalues; $s$ is
known as the signature of the critical point.

The classification of the critical points is as follows.
\begin{itemize}
   \item (3,-3): The three Hessian eigenvalues are all negative. A critical point of this nature represents a maximum. A critical point (3,-3) of the electron density is called \textit{Attractor Critical Point} (ACP).
   \item (3,-1): Two negative and one positive Hessian eigenvalues. This is a saddle point, maximum in two directions and minimum in the other. When the critical point is of the electron density, it is called \textit{Bond Critical Point} (BCP).
   \item (3,+1): One negative and two positive Hessian eigenvalues. This is also a saddle point, but this time is a maximum in one direction and minimum in the rest. The electron density critical point is called \textit{Ring Critical Point} (RCP).
   \item (3,+3): All the three Hessian eigenvalues are positive. This critical point corresponds to a minimum of the field. Critical points of the electron density with this properties are called \textit{Cage Critical Points} (CCP).
\end{itemize}

As a convention, we will use the names ACP, BCP, RCP and CCP for the critical points regardless the density field they are.

Below we present the implemented critical points for different density fields in form of a Table.
%
\begin{table}[hb!]
\begin{center}
\begin{tabular}{||c|c|c||}
\hline
\hline
\textbf{Type} & \multicolumn{2}{|c||}{\textbf{Density Field}} \\
\hline
\hline
 & $\rho$ & LOL ($\gamma$) \\
\hline
ACP & $\checkmark$ & $\checkmark$ \\\hline
BCP & $\checkmark$ & $\checkmark$ \\\hline
RCP & $\checkmark$ & $\boldsymbol{\times}$ \\\hline
CCP & $\checkmark$ & $\boldsymbol{\times}$ \\\hline
\hline
\end{tabular}
\caption{Implemented functions for seeking critical points of different density fields in \DTK{} \dtkversion.}\label{tab:implemseeks}
\end{center}
\end{table}

For the search of critical points, we follow the general strategy proposed by Banerjee \textit{et. al.} \cite{bib:banerjee1985} and Popelier \cite{bib:popelier1994}.

%..............................................................................................
\subsection{Electron density critical points}
%..............................................................................................

Below we enlist some details regarding the searching algorithms for the critical points (CP)
of the electron density. In this section we will use the term \textit{seed} to refer to a
spatial point which serves as a starting point for the search of a critical point. We also
say that two nuclei are \textit{geometrically linked} if the distance between these nuclei
is less than the sum of their respective atomic Covalent radii.

\begin{itemize}
\item[ACP] We start the search for ACPs at each nucleus of the molecule. We use as a seed the 
  coordinates of the nucleus at hand and also a set of seeds displaced a small distance from
  the nucleus. After this, we set other seeds at the midpoints of every pair of 
  geometrically linked pair of nuclei (see above). This is usually enough to find possible
  non-nuclear ACPs (critical points of signature (3,-3), whose position is clearly away
  of any nucleus in the molecule).
\item[BCP] We start the search of BCPs for every couple of geometrically linked nuclei, and the seed
  is set to be the middle point between the pair. After all these points have been used as seeds,
  we look for the largest distance of separation between the geometrically linked atoms,
  and then we perform the search for every couple of atoms that are separated by up to 3.5
  times the maximum bond distance, and we set the seed at the middle point of every one of
  the atoms under this criterion. In this manner, we allow for the Hydrogen bonds
  to be accounted for. We call these BCPs as extended BCPs.
\item[RCP] For each pair of BCPs whose distance of separation is less than the maximum bond distance (as was found for the BCPs search) we set the seed for the search of a RCP to be the middle point of those BCPs.
\item[CCP] For each pair of RCPs in the molecule, if the distance of separation among them is less than two times the maximum bond distance (as found in the search of BCPs), then we set the seed to be the middle point between the two RCPs.
\item [E.S.] (Extended Search) By request, an extended search is done after the first scanning described 
  above. This search uses every non-nuclear ACP, every extended BCP, and all RCPs, and CCPs as 
  points where we construct a set of seeds placed at the centre and vertices of an
  icosahedron. At each one of these seeds, we perform searches of all critical points (except the
  type of the current ACP used for setting the seeds, \textit{i.e.} if we are using an ACP for 
  constructing seeds, then we only search for BCPs, RCPs, and CCPs).
\end{itemize}

%..............................................................................................
\subsection{LOL critical points}
%..............................................................................................

The search for LOL critical points is not finished in version \dtkversion. However, we will keep developing the implementation of these critical points in the near future, and perhaps also for other fields.

\begin{itemize}
\item[ACP] We look for LOL ACPs around every molecule's nucleus. In \DTK{} \dtkversion,
  we set seeds at the centre and vertices of an icosahedron centred at the nucleus position.
\item[BCP] We set the seed for looking LOL BCPs as the middle point between every pair of ACPs found in the molecule.
\end{itemize}

%..............................................................................................
\subsection{Gradient Paths}
%..............................................................................................

A gradient path is defined as a curve that follows the direction of the gradient of the field, and the points that form the gradient paths can be defined through
%
\begin{equation}
   \boldsymbol{r}(s)=\boldsymbol{r}_0(s_1)+\int_{s_1}^{s_2}\nabla F(\boldsymbol{r}(t))dt.
\end{equation}
%
Here $F$ is any wee-defined field, and $\boldsymbol{r}(s)$ is the set of points that belong to the gradient path that passes through $\boldsymbol{r}_0$, and $s$ and $t$ are parametric variables (see Ref. \cite{bib:bader1990book}).

In QTAIM, a gradient path that passes through a BCP is called a bond gradient path or simply a bond path. In \DTK{} \dtkversion, we have implemented the search of bond gradient paths for the electron density. Other fields will be implemented in the future.

The bond paths are integrated using a fifth order Runge-Kutta Dormand-Prince method with a
step size of 0.1 a.u.


%==============================================================================================
\section{Properties along bond paths}
%==============================================================================================

\DTK{} can evaluate any of the implemented scalar fields listed in section \S\ref{sec:availablefields} on the bond paths. The basic algorithm consists of, provided two atoms in a molecule, first finding the BCP, and then integrating the gradient path taking as $\boldsymbol{r}_0$ the BCP. The set of points computed are stored in an array, and then the requested field is evaluated at each of the points belonging to the bond path.

%..............................................................................................
%\subsection{Density Matrix of Order One}
%..............................................................................................

%..............................................................................................
%\subsection{Other fields}
%..............................................................................................









