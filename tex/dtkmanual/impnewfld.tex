%%                     This source code is part of
%% 
%%                   D  E  N  S  T  O  O  L  K  I  T
%% 
%%                          VERSION: 1.0.0
%% 
%%              Contributors: Juan Manuel Solano-Altamirano
%%                            Julio Manuel Hernandez-Perez
%%         Copyright (c) 2013-2015, Juan Manuel Solano-Altamirano
%%                                  <jmsolanoalt@gmail.com>
%% 
%%  -------------------------------------------------------------------
%%  Copyright (c) 2013-2015 Juan Manuel Solano Altamirano.
%%  Permission is granted to copy, distribute and/or modify this document
%%  under the terms of the GNU Free Documentation License, Version 1.3
%%  or any later version published by the Free Software Foundation;
%%  with no Invariant Sections, no Front-Cover Texts, and no Back-Cover Texts.
%%  A copy of the license is included in the section entitled "GNU
%%  Free Documentation License".
%%  ---------------------------------------------------------------------
%% 
%%  If you want to redistribute modifications of the suite, please
%%  consider to include your modifications in our official release.
%%  We will be pleased to consider the inclusion of your contributions.
%%  within the official distribution. Please keep in mind that
%%  scientific software is very special, and version control is 
%%  crucial for tracing bugs. If in despite of this you distribute
%%  your modified version, please do not call it DensToolKit.
%% 
%%  If you find DensToolKit useful, we humbly ask that you cite
%%  the paper(s) on the package --- you can find them on the top
%%  README file.
%% 
%% ************************************************************************
%%%%

%**********************************************************************************************
%**********************************************************************************************
\chapter{Implementing a new field and developer's documentation}
%**********************************************************************************************
%**********************************************************************************************

Implementing a new field in \DTK{} is an easy task. The file \texttt{src/\-com\-mon/\-cust\-fld-wfn\-class.\-cpp}
contains the implementation of a dummy field, which can be easily changed in order to implement
the new field.

\begin{figure}
\lstinputlisting[language=C++, firstline=4, lastline=64]{custfld-wfnclass.cpp}
\caption{Implementation of the scalar dummy field $\rho^2$, and the vector dummy field
$\nabla\rho/\rho$.}
\label{fig:codecustfld}
\end{figure}

For instance, in Fig. \ref{fig:codecustfld} we show how to implement the scalar field $\rho^2$
and the vector field $\nabla\rho/\rho$. The functions evalCustomScalarField and evalCustomVectorField
are already integrated with \DTK's programs. Therefore, computing new user-implemented fields
is, after re-compilation of the suite, almost straightforward. 

In version \dtkversion, this easy implementation of new fields is limited to the type of
new field. If the new field is derivative of any combination of the already implemented fields,
then the implementation is almost trivial. Otherwise, it requires a bit more time and effort to
implement it.

More information about the complete set of functions of the gaussWaveFunc class can be found by
installing doxygen in your system. After this, in \texttt{/top/src} type
\begin{lstlisting}
$make develdocs
\end{lstlisting}
Here we have assumed that \texttt{/top} is the directory where the source code of \DTK{} is 
located. After executing the above command, a file named \texttt{/top/src/devdoc/html/in\-dex.\-html}
should be present. The html file has the developer's documentation created by doxygen.

